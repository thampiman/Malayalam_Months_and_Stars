\documentclass[10pt,twoside]{article}
\usepackage[pdftex]{graphicx}
\usepackage{url}
\usepackage{mdwlist}
\usepackage{hyperref}
\usepackage{amssymb}
\usepackage{amsmath}
\usepackage{mathtools}
\usepackage{lscape}
\usepackage{subfigure}
\usepackage{multirow}
\newcommand{\doctitle}{%
Malayalam Months and Stars}

\pagestyle{myheadings}
\markboth{\hfill\doctitle}{\doctitle\hfill}

\bibliographystyle{IEEEtran}

\addtolength{\textwidth}{1.00in}
\addtolength{\textheight}{1.00in}
\addtolength{\evensidemargin}{-1.00in}
\addtolength{\oddsidemargin}{-0.00in}
\addtolength{\topmargin}{-.50in}

\hyphenation{in-de-pen-dent}

\title{\textbf{\doctitle}\\
For Dummies}

\author{Amma}

\begin{document}

\thispagestyle{empty}

\maketitle

\section{Overview}
A new Malayalam year starts in \verb|Chingam| (Aug 15). As of today, we are in the year \verb|1188|. The following table summarises all the Malayalam months giving corresponding reference dates from the Gregorian calendar.

\begin{table}[!h]
\centering
\begin{tabular}{| c | c | c | c |}
\hline
AUG 15 & SEP 15 & OCT 15 & NOV 15 \\ 
\verb|Chingam| & \verb|Kanni| & \verb|Thullam| & \verb|Vrishchikam| \\ \hline
DEC 15 & JAN 15 & FEB 15 & MAR 15 \\ 
\verb|Dhanu| & \verb|Makaram| & \verb|Kumbham| & \verb|Meenam| \\ \hline
APR 15 & MAY 15 & JUN 15 & JUL 15 \\ 
\verb|Medam| & \verb|Edavam| & \verb|Midhunam| & \verb|Karkidakam| \\ \hline
\end{tabular}
\label{table:overview}
\end{table}

\section{Stars}
There are 27 Malayalam stars and they are:
\begin{enumerate}
\item \verb|Aswathy|
\item \verb|Bharani|
\item \verb|Karthika|
\item \verb|Rohini|
\item \verb|Makayiram|
\item \verb|Thiruvathira|
\item \verb|Punartham|
\item \verb|Pooyam|
\item \verb|Ayilyam|
\item \verb|Makam|
\item \verb|Pooram|
\item \verb|Uthram|
\item \verb|Atham|
\item \verb|Chittira|
\item \verb|Chothi|
\item \verb|Vishakam|
\item \verb|Anizham|
\item \verb|Thrikketta|
\item \verb|Moolam|
\item \verb|Pooradam|
\item \verb|Uthradom|
\item \verb|Thiruvonam|
\item \verb|Avittam|
\item \verb|Chathayam|
\item \verb|Pooruruttathi|
\item \verb|Uthrattathi|
\item \verb|Revathi|
\end{enumerate}

\section{Important Days and Festivals}


\end{document}
 