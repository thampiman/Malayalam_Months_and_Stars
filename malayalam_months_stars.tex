\documentclass[10pt,twoside]{article}
\usepackage[pdftex]{graphicx}
\usepackage{url}
\usepackage{mdwlist}
\usepackage{hyperref}
\usepackage{amssymb}
\usepackage{amsmath}
\usepackage{mathtools}
\usepackage{lscape}
\usepackage{subfigure}
\usepackage{multirow}
\newcommand{\doctitle}{%
Malayalam Months and Stars}

\pagestyle{myheadings}
\markboth{\hfill\doctitle}{\doctitle\hfill}

\bibliographystyle{IEEEtran}

\addtolength{\textwidth}{1.00in}
\addtolength{\textheight}{1.00in}
\addtolength{\evensidemargin}{-1.00in}
\addtolength{\oddsidemargin}{-0.00in}
\addtolength{\topmargin}{-.50in}

\hyphenation{in-de-pen-dent}

\title{\textbf{\doctitle}\\
For Dummies}

\author{Amma}

\begin{document}

\thispagestyle{empty}

\maketitle

\section{Overview}
A new Malayalam year starts in \verb|Chingam| (around Aug 15). As of today, we are in the year \verb|1188|. The following table summarises all the Malayalam months giving corresponding reference dates from the Gregorian calendar.

\begin{table}[!h]
\centering
\begin{tabular}{| c | c | c | c |}
\hline
$\sim$ AUG 15 & $\sim$ SEP 15 & $\sim$ OCT 15 & $\sim$ NOV 15 \\ 
\verb|Chingam| & \verb|Kanni| & \verb|Thullam| & \verb|Vrishchikam| \\ \hline
$\sim$ DEC 15 & $\sim$ JAN 15 & $\sim$ FEB 15 & $\sim$ MAR 15 \\ 
\verb|Dhanu| & \verb|Makaram| & \verb|Kumbham| & \verb|Meenam| \\ \hline
$\sim$ APR 15 & $\sim$ MAY 15 & $\sim$ JUN 15 & $\sim$ JUL 15 \\ 
\verb|Medam| & \verb|Edavam| & \verb|Midhunam| & \verb|Karkidakam| \\ \hline
\end{tabular}
\label{table:overview}
\end{table}

\section{Stars}
There are 27 Malayalam stars and they are:
\begin{enumerate}
\item \verb|Aswathy|
\item \verb|Bharani|
\item \verb|Karthika|
\item \verb|Rohini|
\item \verb|Makayiram|
\item \verb|Thiruvathira|
\item \verb|Punartham|
\item \verb|Pooyam|
\item \verb|Ayilyam|
\item \verb|Makam|
\item \verb|Pooram|
\item \verb|Uthram|
\item \verb|Atham|
\item \verb|Chittira|
\item \verb|Chothi|
\item \verb|Vishakam|
\item \verb|Anizham|
\item \verb|Thrikketta|
\item \verb|Moolam|
\item \verb|Pooradam|
\item \verb|Uthradom|
\item \verb|Thiruvonam|
\item \verb|Avittam|
\item \verb|Chathayam|
\item \verb|Pooruruttathi|
\item \verb|Uthrattathi|
\item \verb|Revathi|
\end{enumerate}

There are only 27 stars whereas there are 30, 31 or sometimes even 32 days in one Malayalam month. So some stars come twice in a month. For your piranaal, if your star comes one more time in the month, the second time is considered your actual piranaal.

\section{Important Days and Festivals}
\begin{enumerate}
\item Onam is in the month of \verb|Chingam|. We start putting pookalam from the \verb|Atham| day for ten days. After ten days, Onam is celebrated on \verb|Thiruvonam| day of \verb|Chingam| month. \verb|Uthradom| is considered to be first Onam. \verb|Thiruvonam| is the second Onam and the actual Onam. \verb|Avittam| is the third Onam. \verb|Chathayam| is the fourth Onam and also the birthday of Sree Narayana Guru.
\item Vishu is the first (or sometimes $2^{nd}$) of the Malayalam month \verb|Medam| and it falls either on the $14^{th}$ (mostly $14^{th}$) or $15^{th}$ of April. Vishu is considered to be our New Year when we keep Kani (beautiful things in front of Lord Krishna).
\item Ramayana Maasam is observed in the month of \verb|Karkidakam|. After the rains, it is generally considered the less bountiful (panja maasam) month. To bring prosperity into our lives, we read the Ramayana. Ladies keep dasapushpam in their hair and people take Ayurvedic treatments to rejuvenate their bodies. Literally we are saying dasapushpam but actually several of them are Ayurvedic herbs and not flowers. The Malayalam names of these herbs are:
\begin{enumerate}
\item Karuka 
\item Vishnukranti
\item Poovam Kurunnila
\item Thiruthali
\item Muyalcheviyan
\item Mukkutti
\item Kayyonni
\item Nilappana
\item Cheroola
\item Uzhinja
\end{enumerate} 
These are all famous Ayurvedic herbs used in preparing many types of Ayurvedic medicines and by just wearing them on the hair for some time may also be helpful in some ways.
\item Navaratri (Dussehra) comes in the \verb|Kanni| maasam. On rare occasions, it comes in \verb|Thulam| maasam also. It is clearly marked in the Malayalam calendar if you search in the months of September and October. Books are kept for pooja on the previous evening of Mahanavami. On Mahanavami day,  we dont read or write and generally take a complete break from studies or work. Vijayadashami day, we worship Goddess Saraswati, write \emph{Harisri Ganapathaye Namah} in rice or sand and restart a year of hard work.
\item Diwali falls usually in the \verb|Thulam| maasam. This also is clearly marked in the Malayalam calendar. Diwali is usually determined by the phase of the moon and it is New Moon (Karatha Vaavu) on Diwali.
\item Makara vilakku (in Sabarimala) and Pongal fall on the $1^{st}$ of \verb|Makara| maasam which is usually the $15^{th}$ of January.
\item Sivarathri is in \verb|Kumbha| maasam which will be between the $15^{th}$ of February and $15^{th}$ of March.
\item Chottanikkara Makam is also in \verb|Kumbha| maasam and falls on the \verb|Makam| star in this month. It is also TVM Ammumma's piranaal.
\item Achan's and Arun's piranaal fall in \verb|Medam| month which is between the $15^{th}$ of April and $15^{th}$ of May. Achan's star is \verb|Makam| and Arun's is \verb|Anizham|. The famous Trichur Pooram starts on Achan's piranaal.
\item Since Sneha's Achan's brithday is on June 1st, his piranaal should be in \verb|Edavam| month which is between May $15^{th}$ and June $15^{th}$. His star is \verb|Pooyam|.
\item Ajay's piranaal is in the month of \verb|Chingam| and his star is \verb|Anizham|. It usually is around Onam time. 
\item Sneha's Amma's piranaal is in \verb|Thulam| month which is between October $15^{th}$ and November $15^{th}$. Her star is \verb|Thiruvonam|.
\item The Sabarimala season starts on the $1^{st}$ of \verb|Vrishchikam| month which is around the $15^{th}$ of November.
\item Sneha's piranaal is in \verb|Vrishchikam| month and the star is \verb|Aswathy|. EKM Ammumma's piranaal also is in \verb|Vrishchikam| month and her star is \verb|Pooyam|.
\item My piranaal falls in the month of \verb|Dhanu| (between $15^{th}$ December and $15^{th}$ January) and the star is \verb|Karthika|. Shibu's piranaal also is in \verb|Dhanu| and his star is \verb|Uthrattathi|. Meera's piranaal should be in \verb|Makara| maasam as I remember it to be in the beginning of February. We have to confirm her naalu.
\item The festival called Thiruvathira also comes in the month of \verb|Dhanu| and is celebrated when the star \verb|Thiruvathira| comes. This is supposed to be Lord Shiva's piranaal when Goddess Parvati observes a fast for the well being of her husband. Women take fasts, not taking any rice items. They have only wheat items and make some special dishes like puzhukku and also koova (arrowroot) kurukkiyathu. The day before \verb|Thiruvathira| is \verb|Makayiram| and many start their fast on that day itself. \verb|Makayiram| fast is for our children and the \verb|Thiruvathira| fast is for your husband.
\end{enumerate}

\end{document}
 